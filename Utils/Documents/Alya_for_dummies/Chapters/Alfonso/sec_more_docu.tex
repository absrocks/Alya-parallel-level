
There are some other ``help sources'' for Alya. First of all you've the official Alya documentation \footnote{\nolinkurl{http://bsccase02.bsc.es/alya/} } were you will find some human-written documentation, and some documentation automatically compiled from the code. Also you'll see some very recommendable tutorials related with running and postprocessing an Alya case.

Another new, not very exploited until today, but very aprofitable way of resolving Alya issues is the mailing list\footnote{\nolinkurl{bsc-alya.bsc.es}, to be added write a line to \nolinkurl{mariano.vazquez@bsc.es} or \nolinkurl{antoni.artigues@bsc.es} }. As it is a new media in which Alya users and developers can communicate, but it can be potentially very congested, there are a few Alya mailing-list ``etiquette rules'' to follow:
\begin{itemize}
	\item Always write in English. The are people from many nationalities using Alya.
	\item In the subject, use tags in capital letters between square brackets to identify the topics. Try not to make up new tags. Some already used and proposed tags are:
	\begin{itemize}
		\item {[PERFORMANCE]}
		\item {[POSTPROCESS]}
		\item {[METHODS]}
		\item {[SOLIDZ/EXMEDI/NASTIN/NASTAL]}
		\item {[NEWFEATURE]}																									
	\end{itemize}
		\item Try not to ask things that are already written in the documentation.
\end{itemize}

So, with those two things, and this guide should be enough to introduce yourself to Alya.