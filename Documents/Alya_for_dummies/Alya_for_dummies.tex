%
% Alya_for_dummies: main file
%

\documentclass[11pt,a4paper,oneside]{book}

\usepackage{amsmath,amsfonts,amssymb}
\usepackage{graphicx,wrapfig}

\usepackage{algpseudocode}
\usepackage{algorithm}
\usepackage{tikz}
\usetikzlibrary{trees}
\usetikzlibrary{arrows, positioning, shapes.geometric}
\usetikzlibrary{arrows, shapes, snakes, automata, backgrounds, petri}
% Define the layers to draw the diagram
\pgfdeclarelayer{background}
\pgfdeclarelayer{foreground}
\pgfsetlayers{background,main,foreground}

\usepackage[hidelinks]{hyperref}
\usepackage{caption}
\usepackage{subcaption}

%Define boxes to write code
\usepackage{xcolor}
\usepackage{listings}
\DeclareCaptionFont{white}{\color{white}}
\DeclareCaptionFormat{listing}{%
	\parbox{\textwidth}{\colorbox{gray}{\parbox{\textwidth}{#1#2#3}}\vskip-4pt}}
\captionsetup[lstlisting]{format=listing,labelfont=white,textfont=white}
\lstset{frame=lrb,xleftmargin=\fboxsep,xrightmargin=-\fboxsep,breaklines=true}



\title{\bf Alya for developers \\for dummies by dummies}

\author{CASE Ph.D students}

\date{\today}

\begin{document}
	
%\maketitle

\thispagestyle{empty}
\begin{figure}[h!]
	\centering
	\includegraphics[width=1.2\textwidth]{./Images/Cover.png}
\end{figure}

\tableofcontents

%Chapter1
%
% Alya_for_dummies
% Chapter 1: 
%

\chapter{Description}

\section{Intention of this document}
This document is intended to be written and readed by all the new and not-that-new Alya users. It contains sections aimed to be used as an \textit{aide-memoire} for the equations, the places where the code is written and why is written in that way.

It also contains some useful metadata about alya, like how to get it, compile it, and information about the pre/run/post-process tools.

This should be a self contained guide, explained as much as possible... for dummies. Or at least for Alya newbies. So if you're an active Alya user, and you have something written that can be helpful, don't doubt in help to complete this file!

\section{What is Alya and what you can do with it}
Alya is a computational tool used to solve multiple physics problems in a coupled way with the Finite Element Method.

The basic explanation of the computational Finite Element Method is founded in the fact that the computer cannot handle continuous spaces. So, having in mind this constriction we go for a more modest approach and suggest to solve the equations that model the phenomena not in a continuum geometry, but in a discretized geometry. In this way, the first thing to do is to split the geometry in several (sometimes hundreds of millions) ``\textit{elements}'' (you'll hear this word A LOT). Each element is formed by the ``\textit{nodes}'' the ``\textit{sides}'' that join the nodes and the ``\textit{faces}'' that join the sides (in the case we've a 3D geometry). So in this moment we've a more humble approach: we're going to solve our equations in the nodes that build our elements; going from a continuous geometrcal space to a discrete one. Something similar happens with time: as far as we know, sadly, it is continuous. A similar approach is used:  we split the time interval  in (generally, but not necessarily) regular portions and we suggest  to solve our problem in those instants of time, even though we know we're ``missing'' information on the middle. In this way, once we've our space and time discretized, we proceed to adapt our continuous equations to a discretized space-time equations, and we proceed to compute the solution in our specified nodes in the space, for our specified time steps.

Alya can solve several phyisical problems modeled as sets of partial different equations. Among others: the compressible and incompressible navier stokes, solid mechanics, turbulence models, chemical reactions, radiation and temperature diffusion among others. This different physical problem are organized inside the code as modules, each user can selectively turn on and off each module to solve that set of PDE's. But, be aware, because if you turn-on a module, you've to specify all the physical properties of it as parameters and boundary conditions.

But Alya not only solve PDEs that represents the physical phenomena of the world, it do it big time, with all the meaning of the phrase. It is coded in parallel, so the problem is distributed between differents processors, using severals of computers as it is one. The way in which Alya does this, is giving a bunch of elements to each processor so, finally, between all the processors they solve the whole geometry.

So if you want, with Alya, you can simulate the world. All of it. Going from the contraction of the muscles in the wing of a butterfly to the airflows that generate tornados. But don't forget that as big as the problem you want to solve is your number of elements, and the parameters and boundary conditions you have to define.


\section{What you will be able to do if you read this document}
As we said before, this document is an introduction to Alya. So, the idea is that when you finish with it, you should be able to:
\begin{itemize}
	\item Compile the software.
	\item Run some simple, pre-built cases.
	\item Know how the code is structurally (des)organized.
	\item Know where the equations are written in the code.
\end{itemize}
So, if you finished with it, you couldn't deal with some of this points, we failed at something. Write us a line or, better, correct it by yourself.






%Chapter2
%
% Alya_for_dummies
% Chapter 1: 
%

\chapter{Some useful hints}


\section{More documentation about Alya}

There are some other ``help sources'' for Alya. First of all you've the official Alya documentation \footnote{\nolinkurl{http://bsccase02.bsc.es/alya/} } were you will find some human-written documentation, and some documentation automatically compiled from the code. Also you'll see some very recommendable tutorials related with running and postprocessing an Alya case.

Another new, not very exploited until today, but very aprofitable way of resolving Alya issues is the mailing list\footnote{\nolinkurl{bsc-alya.bsc.es}, to be added write a line to \nolinkurl{mariano.vazquez@bsc.es} or \nolinkurl{antoni.artigues@bsc.es} }. As it is a new media in which Alya users and developers can communicate, but it can be potentially very congested, there are a few Alya mailing-list ``etiquette rules'' to follow:
\begin{itemize}
	\item Always write in English. The are people from many nationalities using Alya.
	\item In the subject, use tags in capital letters between square brackets to identify the topics. Try not to make up new tags. Some already used and proposed tags are:
	\begin{itemize}
		\item {[PERFORMANCE]}
		\item {[POSTPROCESS]}
		\item {[METHODS]}
		\item {[SOLIDZ/EXMEDI/NASTIN/NASTAL]}
		\item {[NEWFEATURE]}																									
	\end{itemize}
		\item Try not to ask things that are already written in the documentation.
\end{itemize}

So, with those two things, and this guide should be enough to introduce yourself to Alya.

\section{SVN or how to code in a group}
We're about 60 engineers, programmers and physicist working every day in Alya.. This makes a requisite to have a control system to maintain versions of the code, in this case we  use Apache subversion (SVN). To use this, you need svn installed on your computer, and an account in Marenostrum (MN). Let's say that your MN user is "\textless USER\textgreater", so, in this case, some useful svn commands are:

\begin{lstlisting}[label=,caption=Downloading Alya in your local PC (svn checkout)]
svn co svn+ssh://<USER>@mn1.bsc.es/gpfs/projects/bsc21/svnroot/Alya/Trunk <path/to/destination>
\end{lstlisting}

\begin{lstlisting}[label=,caption=Checkout in your marenostrum account]
svn co svn+ssh://<USER>@localhost/gpfs/projects/bsc21/svnroot/Alya/Trunk <path/to/destination>
\end{lstlisting}


\begin{lstlisting}[label=,caption=Updating Alya to the latest version (or an older version)]
svn update [-r <revision_number>]
\end{lstlisting}

\begin{lstlisting}[label=,caption=Information your version]
svn info
\end{lstlisting}

\begin{lstlisting}[label=,caption=Information about who changed the file/directory]
svn blame
\end{lstlisting}

\begin{lstlisting}[label=,caption=Logs of the different modifications in the svn]
svn log
\end{lstlisting}

\begin{lstlisting}[label=,caption=Send a new modification to the repository (commit)]
svn ci
\end{lstlisting}

Almost all the cited commands can be modified adding a path to a file or a directory. So, for example, if you only want to update the svn version of the file called ``solver.f90'', you'll have to type:
%
\begin{lstlisting}[label=,caption=cCommiting just one file]
svn ci solver.f90
\end{lstlisting}

So, now you've Alya in your computer, it's time to compile it.


\section{How to configure and compile Alya}
\subsection{Prerequisites}
Luckily for the users, Alya is almost a selfcontained software. It does not depend on any external library or application. But, of course, you will need a compiler to compile it, particularly, a Fortran and C ones. It has been tested in tons of architectures and operative systems, but, of course, the recommended ones are linux under x86-64 PC. Talking about the compilers, it can be build with the intel and the gcc ones.

We said before that Alya does not have \textit{almost} any dependency. That ``almost'' it because it has one dependency to run in parallel, a mesh partitioning library called metis \footnote{http://glaros.dtc.umn.edu/gkhome/metis/metis/overview}. It solves the very complex geometrical problem of partitioning the mesh in a way to reduce the contact area between sections, in an efficient way. To avoid version problems, this library is included in the Alya repository.

So, let's say you downloaded Alya in the directory ``Alya'' inside the folder of your user. So your absolute path would be: ``/home/USER/Alya''. To compile metis, just go to the directory where the source code is, and type "make":

\begin{lstlisting}[label=,caption=compiling metis]
cd /home/USER/Alya/Thirdparties/metis-4.0
make
\end{lstlisting}

This software is written in C, and that's why you need a C compiler. The rest of Alya is written in modern Fortran.


\subsection{Configuring the compilation (making the Makefile) and compiling Alya}

So now, we are ready to configure and compile Alya. To do this, you are going to work in the directory ``Executables/unix'' inside your Alya instalation folder.

In a first step, you've to create the makefile, in which you specify the modules and services you're going to compile, the kind of optimization you want to use, if you want debugs flags or not; basically exacutable-related things.

The first thing that you've to do is copying the configuration file form the directory ``configure.in'', you'll see three configuration files there: ``config\_ifort.in'' ; ``config\_gfortran.in'' and ``config\_IBM\_xfl.in'', so choose the one that fits your architecture and copy it to the Executables/unix directory with the name config.in:

\belowcaptionskip=-10pt
\begin{lstlisting}[label=copy,caption=Copying the configuration file]
cp configure.in/configure_ifort.in config.in
\end{lstlisting}

Now, if you take a look inside that file, you'll see a first set of line with the wrappers and libraries needed. The next section that you'll see are the optimization flags, which are nicely described there. Remember that if, for example, you uncomment the ``O2'' flag optimization, you'll have to comment the `O1'' optimization.

If you run only 3D problems, activating the ``NDIME PARAMETER'' flag is recommended. It will allow automatic loop unrolling of several loops inside the code improving the performance.

So, in some hypothetical case in which you are running only 3D problems and you'd like to run really fast, your performance flags sections would look something like this:

\belowcaptionskip=-10pt
\begin{lstlisting}[label=opti_flags,caption=fast optimization flags configuration]
#MINUM
#FOPT      = -O1
#MAXIMUM
FOPT      = -O2 -xHost

CSALYA   := $(CSALYA) -DNDIMEPAR

#CSALYA   := $(CSALYA) -ftrapuv -check all -traceback -debug full -warn all,nodec,nointerfaces -fp-stack-check -check all -ansi-alias
\end{lstlisting}
 
So now that you've the characteristics you want for your executable, it's time to build the makefile. Here you also have to specify which modules you want to compile and if you want the debug (-g option) or the release (-x option) version.

\belowcaptionskip=-10pt
\begin{lstlisting}[label=making_makefile,caption=Command to build the makefile]
./configure [-x -g] parall  module1 module2 ...
\end{lstlisting}

Note there that the ``-x'' and ``-g'' are optional if you type none of them, you'll have both versions. Another thing to notice is that, in the way Alya was built, now the ``parall'' module is mandatory and you always have to include it, even though you're not going to use it. So, let's say that we want the release version for the module that solves fluid mechanics, and the module that solves solid mechanics. In that case, you'll have to type:

\belowcaptionskip=-10pt
\begin{lstlisting}[label=making_nastin,caption=Command to build the makefile for fluid and solid mechancis ]
./configure -x nastin solidz parall
\end{lstlisting}

After this point you'll see some stuff in your screen. Take a look at this because it gives you a clue if the Makefile was built in the way you wanted. Here it will specify such things as the compiler used, and the modules that are going to be built.

Once it finishes and you got your shell cursor again, you just type ``make'':

\belowcaptionskip=-10pt
\begin{lstlisting}[label=making,caption=Making Alya]
make [-j n]
\end{lstlisting}

The ``-j'' option allows to compile the code in parallel, and ``n'' specifies the number of cores that are going to be used. It has been tested and you'll see an almost linear scalability up to 4 cores.


\section{Executing an Alya case}
\subsection{File  of a case}
When you get your first example case in your hands you'll see a directory with several files inside, almost all of them finishing with the ``*.dat'' extension. Something that all the files have in common is the beginning of the file name; that's the case name. What each file have inside is related with the physics of the case and with the way it's going to be solved. But, luckily, all the cases that you'll encounter will have the same type of information inside the files with the same extension.

So, a brief description for this is:

\begin{itemize}
	\item \textbf{\textless case\_name\textgreater .dat} it has information about the time discretization and the modules used 
	\item \textbf{\textless case\_name\textgreater.ker.dat} it contains the properties of the materials, some general options about numerical aspects as mesh division, and general posprocessing variable options.
	\item \textbf{\textless case\_name\textgreater.dom.dat} it has options related to the mesh. FOr example the numer of then nodes, the type of elements used or the rule of integration. It also have links (seen as ``include \textless file\_name \textgreater'') to the geometry file, the boundary file and other optional domain related files.
	\item \textbf{\textless case\_name\textgreater.geo.dat} it has the main information about the mesh: the coordinates of each node and the conectivity of the nodes.
	\item \textbf{\textless case\_name\textgreater.\textless physi\textgreater.dat} here, the \textless physi\textgreater portion depends in the physics selected to solve. For example to solve the Navier Stokes incombresible it should be an ``*nsi.dat'' file. Inside of it you'll find options related with the physic of that particular set of equations to solve, and the solver used to obtain the results.
	\item \textbf{\textless case\_name\textgreater.post.alyadat} Yep, it's OK, its ``*.alyadat'' don't ask me why. It have information about the way the postprocess it's going to be done.
\end{itemize}

So basically, those are the files you'll encounter. Now, let's solve the thing.

\subsection{Running the case}
Oh, this is the easy part. Just get inside of the directory of your case, and write the path to the executable, using the problem name as input:

\belowcaptionskip=-10pt
\begin{lstlisting}[label=making,caption=Making Alya]
[mpriun -n nc] /path/to/Alya/Executables/unix/Alya.x <problem_name>
\end{lstlisting}

If you want to run the case in parallel, make sure to execute it with the ``mpirun'' command. If you don't specify the number of cores, it's going to use all the cores available.

\section{User pre-run-post tools}
\subsection{The mesher}
An GNU-licensed easy-to-use mesher that allows you to build simple geometries and meshes to run with Alya is Gmsh \footnote{http://gmsh.info/}

\subsection{ Real time convergence visualzation tools}
There are a few utilities that  you can use to see how your case is going. This tools, as the other home-made tools, are locate in ``Alya/Utils/user''. Here you will see the ``alya-all-\textless phy \textgreater'' commands, where \textless phy \textgreater can be replaced by nsi, nsa, sld (etc.) depending if you are solving Navier Stokes incompresible, Navier Stokes compresible or solid mechanics respectively, among others.

To use them, is pretty simple, just go to the directory in were you're running the case and type the path to the tool, and after that, the case name.

\belowcaptionskip=-10pt
\begin{lstlisting}[label=alya-all,caption=Visualizing the convergence]
[mpriun -n nc] /path/to/Alya/Utils/user/alya-all-<phy> <problem_name>
\end{lstlisting}

This utilities read the ``*cvg'' files, that are written on-the-fly, and plot the graphs with gnuplot\footnote{gnuplot.info}. So,  a good thing about this tools is that they let you see live how your simulation is performing.

\subsection{Postprocessing the results}
So, now that you ran your case, and have all the ``*.post.alyabin'' files (alya binary results files) you need a way to get some software to read them and see those nice good looking colours that you were struggling for. For that, there is also a tool; it's called ``alya2pos'' and it's located in ``Alya/Utils/user/alya2pos''. The only thing is that you have to compile it. For that some generous soul built a script that is located in ``Alya/Executables/unix'' and is called ``alya2pos-compile.sh'', but before executing it you'll have to give it execution permissions:

\belowcaptionskip=-10pt
\begin{lstlisting}[label=alya2pos-compile,caption=Compiling the postprocessing software]
chmod 777 alya2pos-compile.sh
./alya2pos-compile.sh
\end{lstlisting}

So now, as always, you'll have to execute the postprocessing software inside the  problem case directory, using the problem name as attribute for the software:

\belowcaptionskip=-10pt
\begin{lstlisting}[label=alya2pos,caption=Postprocessing the results]
/path/to/Alya/Utils/user/alya2pos/alya2pos.x <problem_name>
\end{lstlisting}

This software reads the \textless problem\_name \textgreater file where it's indicated the output format.

\subsection{Visualizing the results}
A frequently used GNU-licensed visualizing tool is Paraview \footnote{paraview.org}. This software can read ensight files which is one of the possible outputs of the alya2pos software. So here, what you've to do is going to File->open and select the ``\textless problem\_name\textgreater.ensi.case'' file.


\section{The debugger}
\subsection{Debugging by hand}

It is not the most recomended option but sometimes you need it. When
you do it in parallel one option is to use write(kfl\_paral+500) ...
. Raul suggested a nicer option: write({*},{*}) and then you submit
with mpirun -np 5 -outfile-pattern=output.\%r Alya.x 
case-000. The interesting thing is: -outfile-pattern=output.\%r .
With this you get output.0 for proc 0, output1 for proc 1, etc.
There exist also: -errfile-pattern=error.\%r 

\subsection{Debugging with gdb}

This works for small number of parallel processes.
\begin{enumerate}
\item bsub -Is -q interactive -W 1:00 /bin/bash
\item mpirun -n 3 -env DISPLAY \$DISPLAY xterm -e gdb -ex run --args Alya.g
cavtri03\end{enumerate}

\subsection{Debugging with totalview}

First logging to nodes 4 or 5. ssh -Y login5. Then submit the following
run.

\begin{lstlisting}
#!/bin/bash 
#BSUB -n 32 
#BSUB -oo output_32_%J.out 
#BSUB -eo output_32_%J.err 
#BSUB -R"span[ptile=16]" 
#BSUB -J totalview 
#BSUB -x 
#BSUB -W 00:30 
#BSUB -q x11
module purge 
module load impi totalview 
export PATH=$PWD:$PATH 
totalview
\end{lstlisting}


You need to create in the folder where you are going to submit the
run an additional file named mpiexec with:
\begin{lstlisting}
#!/bin/bash 
mpirun $@
\end{lstlisting}


and give it execute rights (chmod a+x mpiexec).

There is more information in the Marenostrum users guide. 

When its starts chose a parallel debugging session and then select
MPI-hydra.




%Chapter3
%
% Alya_for_dummies
% Chapter 1: 
%

\chapter{The code}

\section{The acronyms and names used}
Blah

\section{Directories and file structure}
Blah

\section{Code structure: kernel, services and modules}
%\noindent %%%%%%%%%%%%%%%%%%%%%%%%%%%%%%%%%%%%%%%%%%%%%
\newpage
\subsection{Diagram} %%%%%%%%%%%%%%%
\textsf{}

\tikzstyle{every node}=[draw=black,thick,anchor=west]
\tikzstyle{selected}=[draw=red,fill=red!30]
\tikzstyle{optional}=[dashed,fill=gray!50]
\begin{tikzpicture}[%
  grow via three points={one child      at (0.5,-0.7) and
                                      two children at (0.5,-0.7) and (0.5,-1.4)},
  edge from parent path={(\tikzparentnode.south) |- (\tikzchildnode.west)}]
  %
  %\node [block]                              (b01) {---};
  \node [selected](b02) {Alya} 
    child { node {Turnon}}		
    child { node {Iniunk}}
    child 
    { 
      node [selected] {Times}
      child { node {Timste}}
      child 
      { 
        node [selected] {Blocks}
        child { node {Begste}}
        child 
        { 
          node [selected] {Coupling}
          child { node {Doiter}}
          child { node {Concou}}
        }
        child [missing] {}				
        child [missing] {}				
        child { node {Conblk}}
      }
      child [missing] {}				
      child [missing] {}				
      child [missing] {}				
      child [missing] {}				
      child [missing] {}							
      child [missing] {}							
      child { node {Endste}}
    }
    child [missing] {}				
    child [missing] {}				
    child [missing] {}				
    child [missing] {}				
    child [missing] {}				
    child [missing] {}	    
    child [missing] {}	    
    child [missing] {}				
    child [missing] {}				
    child { node {}};
\end{tikzpicture}


%  !   + current_code                                      %___________current_task 
%  !   |_Alya                                       ______|_____
%  !     |_call Turnon()                            ITASK_TURNON 02  
%  !     |_call Iniunk()                            ITASK_INIUNK 03 
%  !     |_time: do while
%  !       |_call Timste()                          ITASK_TIMSTE 04 
%  !       |_do 
%  !       | |_call Begste()                        ITASK_BEGSTE 05 
%  !       |    |_block: do while                                     _
%  !       |       |_coupling_modules: do while                      / TASK_BEGITE  14 
%  !       |       | |_call Doiter()                ITASK_DOITER 06-|  
%  !       |       | |_call Concou()                ITASK_CONCOU 07  \_ITASK_ENDITE 15 
%  !       |       |_call Conblk()                  ITASK_CONBLK 08 
%  !       |       |_call Newmsh()                  ITASK_NEWMSH 09 
%  !       |_call Endste()                          ITASK_ENDSTE 10 
%  !          __
%  ! BLOCK 3_   | 
%  !   1 X   |  |--current_block  -> CPLNG%blocks_list 
%  !   2 Y Z |  |     
%  !   3 W  _|-----current_module -> CPLNG%moduls_list 
%  ! END_BLOCK__|

\begin{algorithm}
\caption{Alya}
%
\begin{algorithmic}[1]
  %\Start
  \State \Call{Turnon}{ } \Comment{ITASK\_TURNON 02}
  \State \Call{Iniunk}{ } \Comment{ITASK\_INIUNK 03}
  %\End
  \While{Times}
  \State \Call{Timste}{ } \Comment{ITASK\_TIMSTE 04}
  \While{Blocks}
  \State \Call{Begste}{ } \Comment{ITASK\_BEGSTE 05}
  \While{Coupling}
  \State \Call{Doiter}{ } \Comment{ITASK\_DOITER 06}
  \State \Call{Concou}{ } \Comment{ITASK\_CONCOU 07}
  \EndWhile
  \State \Call{Conblk}{ } \Comment{ITASK\_CONBLK 08}
  \EndWhile
  \State \Call{Endste}{ } \Comment{ITASK\_ENDSTE 10}
  \EndWhile
\end{algorithmic}
%
\end{algorithm}


\begin{algorithm}
\caption{Blocks}
%\label{your label for references later in your document}
%
\begin{algorithmic}[1]
  \State BLOCK 3     %_| 
  \State 1 X          %|  |--current_block  -> CPLNG%blocks_list 
  \State 2 Y Z        %|  |     
  \State 3 W          %|-----current_module -> CPLNG%moduls_list 
  \State END\_BLOCK  %_|
\end{algorithmic}
%
\end{algorithm}



%\begin{algorithm}
%\caption{Begste}
%%\label{your label for references later in your document}
%%
%\begin{algorithmic}[1]
%  \State BLOCK 3     %_| 
%  \State 1 X          %|  |--current_block  -> CPLNG%blocks_list 
%  \State 2 Y Z        %|  |     
%  \State 3 W          %|-----current_module -> CPLNG%moduls_list 
%  \State END\_BLOCK  %_|
%\end{algorithmic}
%%
%\end{algorithm}



%\begin{algorithm}
%\end{algorithm}

%% Define block styles
%\tikzstyle{decision} = [diamond, draw, fill=blue!20,text width=4.5em, text badly centered, node distance=3cm, inner sep=0pt]
%\tikzstyle{block}     = [rectangle, draw, fill=blue!20, text width=5em, text centered, rounded corners, minimum height=4em]
%\tikzstyle{line}        = [draw, -latex']
%\tikzstyle{cloud}     = [draw, ellipse,fill=red!20, node distance=3cm,minimum height=2em]

%\begin{tikzpicture}[node distance = 2cm, auto]
%    % Place nodes %%%%%%%%%%%%%%%%%%%%%%%%%%%%%%%%%%%%%%%%%%%%%%%%%%%%%%%%%%%%%%%%%%%%%%%%%%%%%%
%    \node [block]                                                                      (init)         {initialize model};
%    \node [cloud,         left of = init]                                         (expert)    {expert};
%    \node [cloud,       right of = init]                                         (system)   {system};
%    \node [block,     below of = init]                                         (identify)   {identify candidate models};
%    \node [block,     below of = identify]                                   (evaluate) {evaluate candidate models};
%    \node [block,         left of = evaluate, node distance=3cm]  (update)    {update model};
%    \node [decision, below of = evaluate]                                  (decide)     {is best candidate better?};
%    \node [block,     below of = decide, node distance=3cm]     (stop)        {stop};
%    % Draw edges %%%%%%%%%%%%%%%%%%%%%%%%%%%%%%%%%%%%%%%%%%%%%%%%%%%%%%%%%%%%%%%%%%%%%%%%%%%%%%
%    \path [line]             (init)          -- (identify);
%    \path [line]             (identify)   -- (evaluate);
%    \path [line]             (evaluate)  -- (decide);
%    \path [line]             (decide)     -| node [near start] {yes} (update);
%    \path [line]             (update)    |- (identify);
%    \path [line]             (decide)     -- node {no}(stop);
%    \path [line,dashed] (expert)     -- (init);
%    \path [line,dashed] (system)    -- (init);
%    \path [line,dashed] (system)    |- (evaluate);
%    %%%%%%%%%%%%%%%%%%%%%%%%%%%%%%%%%%%%%%%%%%%%%%%%%%%%%%%%%%%%%%%%%%%%%%%%%%%%%%%%%%%%%%%%%
%\end{tikzpicture}

% \newpage
%\begin{tikzpicture}[node distance = 2cm, auto]
%    % Place nodes %%%%%%%%%%%%%%%%%%%%%%%%%%%%%%%%%%%%%%%%%%%%%%%%%%%%%%%%%%%%%%%%%%%%%%%%%%%%%%
%    \node [block]                                     (b01)  {Turnon};
%    \node [block,     below of = b01]       (b02)  {Iniunk};
%    \node [block,     below of = b02]       (b03)  {Timste};
%    \node [block,     below of = b03]       (b04)  {Begste};
%    \node [block,     below of = b04]       (b05)  {Doiter};
%    \node [block,     below of = b05]       (b06)  {Concou};
%    \node [block,     below of = b06]       (b07)  {Conblk};
%    \node [block,     below of = b07]       (b08)  {Endste};
%    %%%%%%%%%%%%%%%%%%%%%%%%%%%%%%%%%%%%%%%%%%%%%%%%%%%%%%%%%%%%%%%%%%%%%%%%%%%%%%%%%%%%%%%%%
%\end{tikzpicture}


\section{The loops}
Blah

%Chapter5
%
% Alya_for_dummies
% Chapter 1: 
%

\chapter{Modules and their equations}

%nastin and nastal
%%%%%%%%%%%%%%%%%%%%%%%%%%%%%%%%%%%%%%%%%%%%%%%%%%%%%%%%%%%%%%%%%%%%%%%%%%%%%%%%%%
\section{Compressible and incompressible Navier-Stokes (nastin and nastal)}

%----------------------------------------------------------------------------------% 
\subsubsection{Fluid Dynamics equations}


\textsf{Navier-Stokes Equations} \cite{muller1999low}
%
\label{eq10.XX} 
\begin{eqnarray}
%
\frac{\partial \rho}{\partial t} + \nabla  \cdot (\rho \textbf{u}) &=& 0    
%
\nonumber \\
%
\frac{\partial \rho \textbf{u}}{\partial t} + \mathbf \nabla  \cdot (\rho \textbf{u}\textbf{u}) + \frac{1}{\tilde M^2} \nabla p &=& \textbf{G}    
%
\nonumber \\
%
\frac{\partial \rho E}{\partial t} + \nabla  \cdot (\rho H \textbf{u}) &=& Q 
%
\nonumber \\
%
\textbf{G} &=& \frac{1}{Re_{\infty}} \nabla \cdot \boldsymbol{\tau} + \frac{1}{Fr^2_{\infty}}\rho( -\textbf{e}_r)
%
\nonumber \\
%
\boldsymbol \tau &=& \mu( \nabla u + (\nabla u)^T ) - \frac{2}{3} \mu \nabla \cdot \textbf{u} \textbf{I} 
%
\nonumber \\
%
Q &=& \rho q +
\frac{\gamma}{(\gamma-1) Re_{\infty} Pr_{\infty}} \nabla \cdot ( \kappa \nabla T )  + 
\frac{\tilde M^2}{Re_{\infty}} \nabla \cdot \boldsymbol \tau \cdot \textbf{u}   
% + \frac{\tilde M^2}{Fr^2_{\infty} } \rho(-\textbf{e}_r) \cdot \textbf{u} + 
%
\nonumber \\
E &=& e + \frac{1}{2} \tilde M^2  \textbf{u}^2 = \frac{1}{\gamma-1} T +  \frac{1}{2} \tilde M^2  \textbf{u}^2 
%
%
\nonumber \\
H &=& E + \frac{p}{\rho} = \rho c_p T +  \frac{1}{2} \tilde M^2  \textbf{u}^2 
%
\nonumber \\
\end{eqnarray} 
%
\footnote{
$$
\rho H = 
\rho E + p = 
\rho( e+ 1/2 \textbf{v}^2) + p =
\rho c_p T + 1/2 \rho \textbf{v}^2; 
$$
%
$$
\partial (\rho H) \partial t + \nabla  \cdot (\rho H \textbf{u}) = 
\partial p/ \partial t    -
\partial q/\partial r_i  + 
\partial (  \tau_{ij} v_j  )/\partial r_i  \tilde M^2/Re_{\infty}
$$
%
$$
D( c_p \rho T)/Dt = 
D p/D t 
-\partial q/\partial r_i  - 
\underbrace{  D(  \rho v_j v_j )/Dt }_{   \propto v_j \partial \sigma_{ij} /\partial r_i }  + 
\partial (  (-p\delta_{ij} + \tau_{ij} ) v_j  )/\partial r_i  \tilde M^2/Re_{\infty}
$$
%
}
%
%$$
%\textbf{G} = \frac{1}{Re_{\infty}} \nabla \cdot \boldsymbol{\tau} + \frac{1}{Fr^2_{\infty}}\rho( -\textbf{e}_r)
%$$
%
%$$
%Q = \rho q +
%\frac{\gamma}{(\gamma-1) Re_{\infty} Pr_{\infty}} \nabla \cdot ( \kappa \nabla T )  + 
%\frac{\tilde M^2}{Re_{\infty}} \nabla \cdot \boldsymbol \tau \cdot \textbf{u}  + 
%\frac{\tilde M^2}{Fr^2_{\infty} } \rho(-\textbf{e}_r) \cdot \textbf{u} + 
%$$
%
with 
$\tilde M = \frac{u^{*}_{\infty}}{\sqrt{p^{*}_{\infty}/\rho^{*}_{\infty}}}  = \sqrt{\gamma} M_{\infty}$ low reference Mach number, 
$Re_{\infty} = \frac{\rho_{\infty}^{*} u_{\infty}^{*} L^{*}}{\mu_{\infty}^{*}} $ the Reynolds number, 
$Fr_{\infty}  = \frac{u_{\infty}^{*}}{\sqrt{g^* L^*}}$ the Froude number, 
$Pr_{\infty}  =  \frac{c_p^* \mu^{*}_{\infty}}{\kappa^{*}_{\infty}}$ then Prandtl number, 
$q = q^{*} \frac{ L^{*} \rho^{*}_{\infty}  }{ u^{*}_{\infty} p^{*}_{\infty}   }$ the nondimensional heat release rate, 
$\boldsymbol \tau$ the nondimensional shear stress tensor 
and 
$\boldsymbol \sigma \equiv -p \textbf{I} + \boldsymbol \tau$.  

The nondimensional expression of the total energy per unit mass and 
the total enthalpy are 
$$
E = e + \tilde M^2 \frac{1}{2} \textbf{u}^2 
$$
%
$$
H = E + \frac{p}{\rho}
$$
The nondimensional equations of state for a perfect gas read
$$
p = \rho T 
$$
%
$$
e = \frac{1}{\gamma-1} T
$$
The pressure expressed in terms of the conservative variables is given by 
$$
p = (\gamma-1) [ \rho E - \tilde M^2 \frac{1}{2} \frac{|\rho \textbf{u}|^2}{\rho} ]
$$

\textsf{Conservative form.} 
The Navier-Stokes equations may be recast in the so-called 
\textit{conservation form of NS} system of equations
\footnote{the repeated indices imply summing and 
the comma denotes partial derivatives with respect to the independent variables $x_i$.}
%
\label{eq04.XX}
\begin{equation}
\frac{\partial \mathbf U}{\partial t} + 
\frac{\partial \mathbf F_i}{\partial r_i} + 
\frac{\partial \mathbf G_i}{\partial r_i} = 
\mathbf B
\end{equation}
%
where  
%
$\textbf{U}$, 
$\mathbf F_i \equiv \mathbf F_i(\mathbf U)$, 
$\mathbf G_i \equiv \mathbf G_i( \mathbf U, \partial \mathbf U/\partial r_j)$, 
and 
$\textbf{B}$ are 
the conservation flow varibles, 
convection flux variables, 
diffusion flux variables, and 
source terms, respectively. 
%
\label{eq04.XX}
\begin{eqnarray}
%\nonumber \\
\mathbf U =
 \begin{pmatrix}
  \rho \\
  \rho v_{\alpha} \\
  \rho E \\
 \end{pmatrix},
%
\mathbf F_i = 
 \begin{pmatrix}
 \rho v_i \\
 \rho v_{\alpha} v_i + p \delta_{\alpha i} \\
 \rho E v_i + p v_i \\ 
 \end{pmatrix}, 
%
\mathbf G_i = 
 \begin{pmatrix}
  0 \\
  -\tau_{i \alpha} \\
  -\tau_{i \alpha} v_{\alpha} + q_i\\
 \end{pmatrix}, 
%
\mathbf B = 
 \begin{pmatrix}
  0 \\
  \rho f_{\alpha} \\
  \rho f_{\alpha} v_{\alpha}\\
 \end{pmatrix}
%
\end{eqnarray}
%

\newpage

%solidz
\section{Electrophysiology (exmedi)}

\subsection{Governing equations}

The governing equations for the electrophisology potential model are described in \cite{Vazquez2011}. The form of the electrical activation potential $\phi$ propagation,  is:
%
\begin{equation}
C_m\frac{\partial \phi}{\partial t} = \frac{\partial}{\partial x_i}\left(\frac{D_{ij}}{S_v}\frac{\partial\phi}{\partial x_j}\right) + I_{ion}
%\nonumber
\end{equation}
%
The latin subscripts count the space dimension of the problem. The difusion term is governed by the diffusion tensor $D_{ij}$ which represents the cells spatial properties. The constants $C_m$ and $S_v$ are the membrane capacitance, and the surface-to-volume ratio, terms introduced by the model. The term $I_{ion}$ differ with the activation propagation model used.

\subsection{The different ionic models}

\subsubsection{The Fitzhugh Nagumo model}

In this case the simpler and fast \textsl{Fitrzhug-Nagumo} \cite{FitzHugh1961} (FHN) is used, defining the ionic current as:
%
\begin{eqnarray}
I_{ion}&=& c_1\phi(\phi-c_3)(\phi-1)+c_2W \\
\frac{\partial W}{\partial t}&=&\epsilon(\phi-\gamma W) \nonumber
\end{eqnarray}
%

\subsubsection{The Ten Tusher model}

\subsection{Excitation Contraction Coupling}

The excitation-contraction (EC) coupling is the way in which the fibres contract after the activation potential propagates trough them. In this heart model it is assumed that the active stress is produced only in the direction of the fibre, so the total cauchy stress can be expresed as:
%
\begin{equation}
\boldsymbol{\sigma}=\boldsymbol{\sigma}_{pas}+\sigma_{act}\left(\lambda,\left[Ca^{+2}\right] \right)\boldsymbol{f}\otimes \boldsymbol{f}
\nonumber
\end{equation}
%
where $\sigma_{pas}$ is the passive stress defined in \cite{Lafortune2012} as a function of the four invariants and the strain energy function defined by Holzapfel and Ogden in \cite{holzapfel2009}, $f$ is a unit vector aligned to the fibre, and $\lambda$ is the fiber stretch. The active stress used in this work is the defined by Hunter in  \cite{Nash2000}, where the active tension is a function depending on the value of calcium concentration:

\begin{equation}
\sigma_{act}=\frac{\left[Ca^{+2}\right]^n}{\left[Ca^{+2}\right]^n+C_{50}^n}+\sigma_{max}( 1+\eta(\lambda-1))
\nonumber
\end{equation}

where $C_{50}$ is the value of the intracellular calcium concentration for $50\%$ of $\sigma_{max}$, $n$ is a coefficient that controls the shape of the curve, and $\sigma_{max}$ is the maximum tensile stress generated at $\lambda=1$  \cite{Lafortune2012}. The $\left[Ca^{+2}\right]$ concentration used in this work is the function defined by Hunter in \cite{Hunter1998}.
\newpage
%exmedi

%Coupled problems
%%%%%%%%%%%%%%%%%%%%%%%%%%%%%%%%%%%%%%%%%%%%%%%%%%%%%%%%%%%%%%%%%%%%%%%%%%%%%%%%%%
\section{Coupling}


\subsubsection{Coupled and uncoupled problems} 
%
We assume that these equations may be discretized using some numerical technique by the linear systems \cite{whiteley2011error}
%
\label{eq19.03}
\begin{equation}
A_{11} u_{1} = b_{1}, A_{22} u_{2} = b_{2}
\end{equation}
%
where the first equation corresponds to the first model and 
the second equation to the second model. 

Now suppose we wish to couple the two mathematical models 
that have been discretised by (\ref{eq19.03}) 
%
\label{eq19.04}
\begin{equation}
%
\begin{pmatrix}
A_{11}  & A_{12}  \\
A_{21}  & A_{22}  \\
\end{pmatrix} =
%
\begin{pmatrix}
b_{1}  \\
b_{2}  \\
\end{pmatrix} 
%
\end{equation}
%
where 
$A_{11}$ is a $M \times M$ matrix, 
$A_{12}$ is a $M \times N$ matrix, 
$A_{21}$ is a $N \times M$ matrix and 
$A_{22}$ is a $N \times N$ matrix;  
%
$b_1$ and $u_1$ are  vectors of lengh $M$, 
$b_2$ and $u_2$ are  vectors of lengh $N$. 
%
$A_{12}$ characterises the effect of $u_2$ on the first model and 
$A_{21}$ characterises the effect of $u_1$ on the second model ($A_{ij}$, $u_j \rightarrow  A_{ii}$). 


\subsubsection*{Schur complement} 
%
\label{eq19.05}
\begin{equation}
%
\begin{pmatrix}
\textbf{A} & \textbf{B} \\
\textbf{C} & \textbf{D} \\
\end{pmatrix} 
\begin{pmatrix}
\textbf{x} \\
\textbf{y} \\
\end{pmatrix} =
\begin{pmatrix}
\textbf{f} \\
\textbf{g} \\
\end{pmatrix}
%
\end{equation}
%
$$
\textbf{x} = \textbf{A}^{-1} (\textbf{f} - \textbf{B} \textbf{y}) 
$$
%
\label{eq19.06}
\begin{equation}
%
\underbrace{ (\textbf{D} - \textbf{C}\textbf{A}^{-1}\textbf{B}) }_{ \textbf{S}_{y} }
\textbf{y} 
= 
\underbrace{ \textbf{g} - \textbf{C}\textbf{A}^{-1}\textbf{f} }_{ \textbf{g}_{y} }
%
\end{equation}
%
The matrix $\textbf{S}_{y}$ is called the \textit{Schur complement} system for $\textbf{y}$. 
Once $\textbf{y} (= \textbf{S}_{y}^{-1} \textbf{g}_{y}$) is found, the internal components $\textbf{x}$ can be found by using (\ref{eq19.06}). 
An approximation, 
$
\lambda_{\textbf{y}} \equiv 
\textbf{C} \textbf{x} + \textbf{D}\textbf{y} - \textbf{g} = 
\textbf{S}_y \textbf{y} - \textbf{g}_y
$ 
of the functional representing the normal derivarive coincides with the residual $\textbf{r}$ corresponding 
to the nodes on $\Gamma$ of a \textit{Poisson problem} with a Neumann condition on $\Gamma$ \cite{toselli2005domain}. 
%
%$\textbf{S}_y \textbf{y} = \textbf{g}_y$, 
%
%

\newpage




%Chapter4
%
% Alya_for_dummies
% Chapter 1: 
%

\chapter{Explained cases}

\section{The heart model}
Blah

\subsection{Which are the equations?}
Blah

\subsection{Where are they coded?}
Blah




\end{document}
